\documentclass[openany, 12pt, draft]{book}
\usepackage{lmodern}
\usepackage{amssymb,amsmath}
\usepackage{ifxetex,ifluatex}
\usepackage{fixltx2e} % provides \textsubscript
\ifnum 0\ifxetex 1\fi\ifluatex 1\fi=0 % if pdftex
  \usepackage[T1]{fontenc}
  \usepackage[utf8]{inputenc}
\else % if luatex or xelatex
  \ifxetex
    \usepackage{mathspec}
  \else
    \usepackage{fontspec}
  \fi
  \defaultfontfeatures{Ligatures=TeX,Scale=MatchLowercase}
\fi
% use upquote if available, for straight quotes in verbatim environments
\IfFileExists{upquote.sty}{\usepackage{upquote}}{}
% use microtype if available
\IfFileExists{microtype.sty}{%
\usepackage{microtype}
\UseMicrotypeSet[protrusion]{basicmath} % disable protrusion for tt fonts
}{}
\usepackage{hyperref}
\hypersetup{unicode=true,
            pdftitle={Rose's Thesis\ldots{}},
            pdfauthor={Student number: 2417024},
            pdfborder={0 0 0},
            breaklinks=true}
\urlstyle{same}  % don't use monospace font for urls
\usepackage{natbib}
\bibliographystyle{BES.bst}
\usepackage{longtable,booktabs}
\usepackage{graphicx,grffile}
\makeatletter
\def\maxwidth{\ifdim\Gin@nat@width>\linewidth\linewidth\else\Gin@nat@width\fi}
\def\maxheight{\ifdim\Gin@nat@height>\textheight\textheight\else\Gin@nat@height\fi}
\makeatother
% Scale images if necessary, so that they will not overflow the page
% margins by default, and it is still possible to overwrite the defaults
% using explicit options in \includegraphics[width, height, ...]{}
\setkeys{Gin}{width=\maxwidth,height=\maxheight,keepaspectratio}
\IfFileExists{parskip.sty}{%
\usepackage{parskip}
}{% else
\setlength{\parindent}{0pt}
\setlength{\parskip}{6pt plus 2pt minus 1pt}
}
\setlength{\emergencystretch}{3em}  % prevent overfull lines
\providecommand{\tightlist}{%
  \setlength{\itemsep}{0pt}\setlength{\parskip}{0pt}}
\setcounter{secnumdepth}{5}
% Redefines (sub)paragraphs to behave more like sections
\ifx\paragraph\undefined\else
\let\oldparagraph\paragraph
\renewcommand{\paragraph}[1]{\oldparagraph{#1}\mbox{}}
\fi
\ifx\subparagraph\undefined\else
\let\oldsubparagraph\subparagraph
\renewcommand{\subparagraph}[1]{\oldsubparagraph{#1}\mbox{}}
\fi

%%% Use protect on footnotes to avoid problems with footnotes in titles
\let\rmarkdownfootnote\footnote%
\def\footnote{\protect\rmarkdownfootnote}

%%% Change title format to be more compact
\usepackage{titling}

% Create subtitle command for use in maketitle
\providecommand{\subtitle}[1]{
  \posttitle{
    \begin{center}\large#1\end{center}
    }
}

\setlength{\droptitle}{-2em}

  \title{Rose's Thesis\ldots{}}
    \pretitle{\vspace{\droptitle}\centering\huge}
  \posttitle{\par}
    \author{Student number: 2417024}
    \preauthor{\centering\large\emph}
  \postauthor{\par}
      \predate{\centering\large\emph}
  \postdate{\par}
    \date{Last edited: October 24 2019}

\usepackage{booktabs, amsthm, setspace}
\usepackage[left]{lineno}
\linenumbers
\doublespacing
\setmainfont{Arial}
\setcounter{secnumdepth}{1}

\usepackage[nottoc]{tocbibind}
\bibpunct{(}{)}{;}{a}{}{;}
\setlength{\bibsep}{1em}
\setlength{\bibhang}{1em}
\renewcommand{\bibname}{References}

\makeatletter
\def\thm@space@setup{%
  \thm@preskip=8pt plus 2pt minus 4pt
  \thm@postskip=\thm@preskip
}
\makeatother

\begin{document}
\maketitle

{
\setcounter{tocdepth}{1}
\tableofcontents
}
If I put some blurb here where does that go?

\hypertarget{intro}{%
\chapter{Introduction}\label{intro}}

\begin{quote}
Would some images of genome duplication be helpful?
\end{quote}

Whole-genome duplication (polyploidisation) is an extreme mutation which occurs due to errors during cell division, such as non-disjunction; sister chromatids are not separated during anaphase and so all end up in one daughter cell, leaving the other void of genetic material.
When this occurs during gametogenesis it can lead to whole organisms with more than two complete sets of chromosomes (polyploids).
According to \citet{REFNEEDED} the estimated rate of polyploidisation is one order of magnitude higher than that of standard mutation (10\textsuperscript{-5}).
In humans whole-genome duplication accounts for around 5\% of miscarriages \citep{REFNEEDED}, and only one polyploid mammal has ever been recorded \citep{REFNEEDED}.
But while, in many taxa, this mutation is not well tolerated, the same is not so true of amphibians, fish, fungi and plants.
Polyploidy is exceptionally well tolerated in plants. It's a ubiquitous feature in the lineages of almost all angiosperms \citep{Jiao2011}, and it's been found to have occurred recently (post genera formation) for 35\% of all vascular plants \citep{Wood2009}.
Plant surveys of natural populations over the last century further support this, reporting mean polyploid frequencies of 25\% \citep{REFNEEDED}.

\hypertarget{so-why-do-plants-tolerate-genome-doubling-so-well}{%
\section{So why do plants tolerate genome-doubling so well?}\label{so-why-do-plants-tolerate-genome-doubling-so-well}}

This process of genome-doubling is thought to be fundamental in the diversification of plant species; having been found to coincide with 15\% of angiosperm speciation events, and 31\% of ferns \citep{Wood2009}.
Furthermore, \citet{Jiao2011} showed that whole-genome duplication could be linked with major innovation that lead to diversification of the phyla.
Their phylogenetic analysis brought to light two ancient groups of duplication events (around 319 and 192 MYA).
These first grouping of duplications influenced the diversification of regulatory genes that were integral to seed development and the more recent grouping affected genes that enabled flower development.
Therefore, the authors conclude that these duplications played a major role in the rise of all seed plants, as well as in their later diversification to both gymnosperm and angiosperm phyla.

However, the situation is less clear-cut; according to \citet{Mayrose2011} polyploidisation significantly increases speciation rates of diploids, but new polyploid lines don't further speciate by that same mechanism.
So polyploid speciation rates are smaller in comparison.
Furthermore, polyploid extinction rates are greater than those of diploids.
This Likelihood-based analysis of vascular plants provides clear support for the traditionally popular view that polyploidy most often leads to evolutionary dead ends.
So, what is it then that allows this extreme mutation to succeed and become fixated?
What scenarios and mechanisms combine to allow rare polyploid lines to succeed?
Are the majority of the polyploids we see in natural populations doomed to extinction?
And, can we predict which conditions will lead to their survival?

Despite having been studied for over a century, the factors that drive the success of polyploid establishment in the face of high extinction rates are still unclear.
\citet{Arrigo2012} conclude that polyploids tend to become extinct at the establishment phase due to reproductive disadvantages such as triploid sterility, or limited mate-choice; the latter occurring via diploid pollen-swamping, or delayed flowering.
On the other hand, polyploids are frequently linked with distinct traits such as ``gigas effects'', which describes increased size and plant organs.
Additionally, benefits of polyploidy can include enhanced capabilities for buffering of deleterious mutation (due to increased heterozygosity), hybrid vigour (heterosis), and--somewhat contentiously--reversal of selfing inhibition \citep{Woodhouse2009, Ramsey2014}.
These traits are thought to overcome the reproductive disadvantages of polyploidy and, in those rare cases, make this mutation key to the invasive and adaptive potential of plants. Let's examine these mechanisms in more detail:

\hypertarget{benefits}{%
\section{Benefits}\label{benefits}}

\hypertarget{gigas}{%
\subsection{Gigas-effects}\label{gigas}}

\hypertarget{buffering}{%
\subsection{Genetic buffering}\label{buffering}}

\hypertarget{vigour}{%
\subsection{Hybrid Vigour}\label{vigour}}

\hypertarget{costs}{%
\section{Costs}\label{costs}}

\begin{quote}
Are they realistic? How do they work? Why/when are they important? consider the conditions.
\end{quote}

\hypertarget{pollen-swamping}{%
\subsection{Diploid Pollen-Swamping}\label{pollen-swamping}}

\hypertarget{flowering}{%
\subsection{Delayed Flowering}\label{flowering}}

\hypertarget{sterility}{%
\subsection{Triploid Sterility}\label{sterility}}

\begin{quote}
So what are the core mechanisms?
\end{quote}

\begin{quote}
Or the most suspicious?
\end{quote}

\hypertarget{how-do-these-mechanisms-link}{%
\section{How do these mechanisms link?}\label{how-do-these-mechanisms-link}}

\begin{quote}
Does limited mate-choice set the scene for the evolution of selfing vs out-crossing?
\end{quote}

\begin{quote}
Delayed flowering (cost) is associated with gigas-effects of increased size (benefit). TRADE-OFF.
\end{quote}

\hypertarget{aims}{%
\section{Aims}\label{aims}}

Based on all this pondering - what am I actually going to examine?

\bibliography{book.bib,packages.bib}


\end{document}
