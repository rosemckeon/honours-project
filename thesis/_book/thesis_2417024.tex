\documentclass[openany, 12pt, draft]{book}
\usepackage{lmodern}
\usepackage{amssymb,amsmath}
\usepackage{ifxetex,ifluatex}
\usepackage{fixltx2e} % provides \textsubscript
\ifnum 0\ifxetex 1\fi\ifluatex 1\fi=0 % if pdftex
  \usepackage[T1]{fontenc}
  \usepackage[utf8]{inputenc}
\else % if luatex or xelatex
  \ifxetex
    \usepackage{mathspec}
  \else
    \usepackage{fontspec}
  \fi
  \defaultfontfeatures{Ligatures=TeX,Scale=MatchLowercase}
\fi
% use upquote if available, for straight quotes in verbatim environments
\IfFileExists{upquote.sty}{\usepackage{upquote}}{}
% use microtype if available
\IfFileExists{microtype.sty}{%
\usepackage{microtype}
\UseMicrotypeSet[protrusion]{basicmath} % disable protrusion for tt fonts
}{}
\usepackage{hyperref}
\hypersetup{unicode=true,
            pdftitle={Rose's Thesis\ldots{}},
            pdfauthor={Student number: 2417024},
            pdfborder={0 0 0},
            breaklinks=true}
\urlstyle{same}  % don't use monospace font for urls
\usepackage{natbib}
\bibliographystyle{BES.bst}
\usepackage{longtable,booktabs}
\usepackage{graphicx,grffile}
\makeatletter
\def\maxwidth{\ifdim\Gin@nat@width>\linewidth\linewidth\else\Gin@nat@width\fi}
\def\maxheight{\ifdim\Gin@nat@height>\textheight\textheight\else\Gin@nat@height\fi}
\makeatother
% Scale images if necessary, so that they will not overflow the page
% margins by default, and it is still possible to overwrite the defaults
% using explicit options in \includegraphics[width, height, ...]{}
\setkeys{Gin}{width=\maxwidth,height=\maxheight,keepaspectratio}
\IfFileExists{parskip.sty}{%
\usepackage{parskip}
}{% else
\setlength{\parindent}{0pt}
\setlength{\parskip}{6pt plus 2pt minus 1pt}
}
\setlength{\emergencystretch}{3em}  % prevent overfull lines
\providecommand{\tightlist}{%
  \setlength{\itemsep}{0pt}\setlength{\parskip}{0pt}}
\setcounter{secnumdepth}{5}
% Redefines (sub)paragraphs to behave more like sections
\ifx\paragraph\undefined\else
\let\oldparagraph\paragraph
\renewcommand{\paragraph}[1]{\oldparagraph{#1}\mbox{}}
\fi
\ifx\subparagraph\undefined\else
\let\oldsubparagraph\subparagraph
\renewcommand{\subparagraph}[1]{\oldsubparagraph{#1}\mbox{}}
\fi

%%% Use protect on footnotes to avoid problems with footnotes in titles
\let\rmarkdownfootnote\footnote%
\def\footnote{\protect\rmarkdownfootnote}

%%% Change title format to be more compact
\usepackage{titling}

% Create subtitle command for use in maketitle
\providecommand{\subtitle}[1]{
  \posttitle{
    \begin{center}\large#1\end{center}
    }
}

\setlength{\droptitle}{-2em}

  \title{Rose's Thesis\ldots{}}
    \pretitle{\vspace{\droptitle}\centering\huge}
  \posttitle{\par}
    \author{Student number: 2417024}
    \preauthor{\centering\large\emph}
  \postauthor{\par}
      \predate{\centering\large\emph}
  \postdate{\par}
    \date{Last edited: October 23 2019}

\usepackage{booktabs, amsthm, setspace}
\usepackage[left]{lineno}
\linenumbers
\doublespacing
\setmainfont{Arial}
\setcounter{secnumdepth}{1}

\usepackage[nottoc]{tocbibind}
\bibpunct{(}{)}{;}{a}{}{;}
\setlength{\bibsep}{1em}
\setlength{\bibhang}{1em}
\renewcommand{\bibname}{References}

\makeatletter
\def\thm@space@setup{%
  \thm@preskip=8pt plus 2pt minus 4pt
  \thm@postskip=\thm@preskip
}
\makeatother

\begin{document}
\maketitle

{
\setcounter{tocdepth}{1}
\tableofcontents
}
\chapter{Introduction}\label{intro}

Polyploidy is well tolerated in plants, being a ubiquitous feature in
the lineages of most plant taxa \citep{Ramsey2014}. While they have been
studied for over a century, the factors that drive the success of
polyploid establishment in the face of reproductive disadvantages and
high extinction rates are still unclear. What is it that allows this
extreme mutation to persist and become fixated within a population?

The process of whole-genome duplication is also thought to be
fundamental in the diversification of plant species; having been found
to coincide with around 25\% of plant speciation events
\citep{Wood2009}.

\begin{quote}
Make sure this figure is as correct as can be. Does it need the
breakdown between different taxa?
\end{quote}

However, a study by \citet{Arrigo2012} on the extinction rates of
polyploids has suggested that many of these new lines are evolutionary
dead ends

\begin{quote}
MORE DETAIL on evolutionary dead ends and exact extinction rates.
\end{quote}

\section{Costs}\label{costs}

They conclude that polyploids tend to become extinct at the
establishment phase due to reproductive disadvantages such as limited
mate-choice (by diploid pollen-swamping, or delayed flowering), and
triploid sterility.

\subsection{Diploid Pollen-Swamping}\label{diploid-pollen-swamping}

\subsection{Delayed Flowering}\label{delayed-flowering}

\subsection{Triploid Sterility}\label{triploid-sterility}

\section{Benefits}\label{benefits}

Benefits associated with polyploidy may offset these costs: Polyploids
are frequently linked with distinct traits such as ``gigas effects'',
which include increases in plant organs, reversal of selfing inhibition,
enhanced capabilities for buffering of deleterious mutation (due to
increased heterozygosity), and hybrid vigour (heterosis)
\citep{Woodhouse2009, Ramsey2014}. These traits are thought to overcome
the reproductive disadvantages of polyploidy and instead make this
mutation key to the invasive and adaptive potential of plants,
ultimately shaping broader patterns of plant diversification.

\subsection{Gigas-effects}\label{gigas-effects}

\subsection{Genetic buffering}\label{genetic-buffering}

\subsection{Hybrid Vigour}\label{hybrid-vigour}

\section{So what are the core
mechanisms?}\label{so-what-are-the-core-mechanisms}

\section{Or the most suspicious?}\label{or-the-most-suspicious}

\section{Or how do these mechanisms
link?}\label{or-how-do-these-mechanisms-link}

\begin{quote}
Does limited mate-choice set the scene for the evolution of selfing vs
outcrossing?
\end{quote}

\begin{quote}
Delayed flowering (cost) is associated with gigas-effects of increased
size (benefit). TRADE-OFF.
\end{quote}

\section{Under what different conditions do these mechanisms
work?}\label{under-what-different-conditions-do-these-mechanisms-work}

\bibliography{book.bib,packages.bib}


\end{document}
