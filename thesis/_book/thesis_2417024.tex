\documentclass[openany, 12pt, draft]{book}
\usepackage{lmodern}
\usepackage{amssymb,amsmath}
\usepackage{ifxetex,ifluatex}
\usepackage{fixltx2e} % provides \textsubscript
\ifnum 0\ifxetex 1\fi\ifluatex 1\fi=0 % if pdftex
  \usepackage[T1]{fontenc}
  \usepackage[utf8]{inputenc}
\else % if luatex or xelatex
  \ifxetex
    \usepackage{mathspec}
  \else
    \usepackage{fontspec}
  \fi
  \defaultfontfeatures{Ligatures=TeX,Scale=MatchLowercase}
\fi
% use upquote if available, for straight quotes in verbatim environments
\IfFileExists{upquote.sty}{\usepackage{upquote}}{}
% use microtype if available
\IfFileExists{microtype.sty}{%
\usepackage{microtype}
\UseMicrotypeSet[protrusion]{basicmath} % disable protrusion for tt fonts
}{}
\usepackage{hyperref}
\hypersetup{unicode=true,
            pdftitle={Rose's Thesis\ldots{}},
            pdfauthor={Student number: 2417024},
            pdfborder={0 0 0},
            breaklinks=true}
\urlstyle{same}  % don't use monospace font for urls
\usepackage{natbib}
\bibliographystyle{BES.bst}
\usepackage{longtable,booktabs}
\usepackage{graphicx,grffile}
\makeatletter
\def\maxwidth{\ifdim\Gin@nat@width>\linewidth\linewidth\else\Gin@nat@width\fi}
\def\maxheight{\ifdim\Gin@nat@height>\textheight\textheight\else\Gin@nat@height\fi}
\makeatother
% Scale images if necessary, so that they will not overflow the page
% margins by default, and it is still possible to overwrite the defaults
% using explicit options in \includegraphics[width, height, ...]{}
\setkeys{Gin}{width=\maxwidth,height=\maxheight,keepaspectratio}
\IfFileExists{parskip.sty}{%
\usepackage{parskip}
}{% else
\setlength{\parindent}{0pt}
\setlength{\parskip}{6pt plus 2pt minus 1pt}
}
\setlength{\emergencystretch}{3em}  % prevent overfull lines
\providecommand{\tightlist}{%
  \setlength{\itemsep}{0pt}\setlength{\parskip}{0pt}}
\setcounter{secnumdepth}{5}
% Redefines (sub)paragraphs to behave more like sections
\ifx\paragraph\undefined\else
\let\oldparagraph\paragraph
\renewcommand{\paragraph}[1]{\oldparagraph{#1}\mbox{}}
\fi
\ifx\subparagraph\undefined\else
\let\oldsubparagraph\subparagraph
\renewcommand{\subparagraph}[1]{\oldsubparagraph{#1}\mbox{}}
\fi

%%% Use protect on footnotes to avoid problems with footnotes in titles
\let\rmarkdownfootnote\footnote%
\def\footnote{\protect\rmarkdownfootnote}

%%% Change title format to be more compact
\usepackage{titling}

% Create subtitle command for use in maketitle
\providecommand{\subtitle}[1]{
  \posttitle{
    \begin{center}\large#1\end{center}
    }
}

\setlength{\droptitle}{-2em}

  \title{Rose's Thesis\ldots{}}
    \pretitle{\vspace{\droptitle}\centering\huge}
  \posttitle{\par}
    \author{Student number: 2417024}
    \preauthor{\centering\large\emph}
  \postauthor{\par}
      \predate{\centering\large\emph}
  \postdate{\par}
    \date{Last edited: October 23 2019}

\usepackage{booktabs, amsthm, setspace}
\usepackage[left]{lineno}
\linenumbers
\doublespacing
\setmainfont{Arial}
\setcounter{secnumdepth}{1}

\usepackage[nottoc]{tocbibind}
\bibpunct{(}{)}{;}{a}{}{;}
\setlength{\bibsep}{1em}
\setlength{\bibhang}{1em}
\renewcommand{\bibname}{References}

\makeatletter
\def\thm@space@setup{%
  \thm@preskip=8pt plus 2pt minus 4pt
  \thm@postskip=\thm@preskip
}
\makeatother

\begin{document}
\maketitle

{
\setcounter{tocdepth}{1}
\tableofcontents
}
\chapter{Introduction}\label{intro}

Whole-genome duplication (polyploidisation) occurs due to errors during
cell division, such as non-disjuntion. The result is either single cells
or--when that cell is a gamete--whole organisms that become polyploid,
and so posses more than two complete sets of chromosomes. In many taxa
this mutation is not well tolerated; for instance, in humans
whole-genome duplication accounts for around 5\% of miscarriages
\citep{REFNEEDED}, and only one polyploid mammal has been recorded
\citep{REFNEEDED}. However, the same is not so true of amphibians, fish,
fungi and plants.

\begin{quote}
Image here about Whole-genome duplication / levels of ploidy?
\end{quote}

Polyploidy is exceptionally well tolerated in plants, being a ubiquitous
feature in the lineages of almost all angiosperms \citep{Jiao2011}, and
having occured recently (post genera formation) for 35\% of all vascular
plants \citep{Wood2009}. Furthermore, \citet{Jiao2011} also showed that
whole-genome duplication could be linked with major innovation. Their
phylogenetic analysis brought to light two ancient groups of duplication
events (around 319 and 192 MYA). These events lead to the
diversification of regulatory genes that were integral to seed
development and later to genes that enabled flower development.
Therefore, these duplications contributed to the appearance and success
of all seed plants and angiosperms.

However, despite having been studied for over a century, the factors
that drive the success of polyploid establishment in the face of
reproductive disadvantages and high extinction rates are still unclear.
What is it that allows this extreme mutation to persist and become
fixated within plant communities?

The process of whole-genome duplication is thought to be fundamental in
the diversification of plant species; having been found to coincide with
around 15\% of angiosperm speciation events, and 31\% in ferns
\citep{Wood2009}. However, the situation was later found to be less
clear-cut; according to \citet{Mayrose2011} polyploidisation is critical
in increasing speciation rates of diploids, but new polyploid lines
don't further speciate by that same mechanism, and so their speciation
rates are smaller in comparison. Furthermore, their extinction rates are
greater than those of diploids. This Liklihood-based analysis of
vascualr plants provided the first quantitative support for the
traditionally popular view that polyploidy most often leads to
evolutionary dead ends.

\section{Costs}\label{costs}

\citet{Arrigo2012} conclude that polyploids tend to become extinct at
the establishment phase due to reproductive disadvantages such as
triploid sterility, or limited mate-choice; the latter occuring via
diploid pollen-swamping, or delayed flowering. By exploring each
mechanism in more detail, we can start to get a feel for how they work,
the conditions that will cause them to be important and, ultimately,
whether or not they are realistic.

\begin{quote}
Are they realistic? How do they work? Why/when are they important?
consider the conditions.
\end{quote}

\subsection{Diploid Pollen-Swamping}\label{pollen-swamping}

\subsection{Delayed Flowering}\label{flowering}

\subsection{Triploid Sterility}\label{sterility}

\section{Benefits}\label{benefits}

Benefits associated with polyploidy may offset these costs: Polyploids
are frequently linked with distinct traits such as ``gigas effects'',
which include increases in plant organs, reversal of selfing inhibition,
enhanced capabilities for buffering of deleterious mutation (due to
increased heterozygosity), and hybrid vigour (heterosis)
\citep{Woodhouse2009, Ramsey2014}. These traits are thought to overcome
the reproductive disadvantages of polyploidy and instead make this
mutation key to the invasive and adaptive potential of plants,
ultimately shaping broader patterns of plant diversification.

\subsection{Gigas-effects}\label{gigas}

\subsection{Genetic buffering}\label{buffering}

\subsection{Hybrid Vigour}\label{vigour}

\begin{quote}
So what are the core mechanisms?
\end{quote}

\begin{quote}
Or the most suspicious?
\end{quote}

\section{How do these mechanisms
link?}\label{how-do-these-mechanisms-link}

\begin{quote}
Does limited mate-choice set the scene for the evolution of selfing vs
outcrossing?
\end{quote}

\begin{quote}
Delayed flowering (cost) is associated with gigas-effects of increased
size (benefit). TRADE-OFF.
\end{quote}

\bibliography{book.bib,packages.bib}


\end{document}
