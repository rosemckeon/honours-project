\documentclass[openany, 12pt, draft]{book}
\usepackage{lmodern}
\usepackage{amssymb,amsmath}
\usepackage{ifxetex,ifluatex}
\usepackage{fixltx2e} % provides \textsubscript
\ifnum 0\ifxetex 1\fi\ifluatex 1\fi=0 % if pdftex
  \usepackage[T1]{fontenc}
  \usepackage[utf8]{inputenc}
\else % if luatex or xelatex
  \ifxetex
    \usepackage{mathspec}
  \else
    \usepackage{fontspec}
  \fi
  \defaultfontfeatures{Ligatures=TeX,Scale=MatchLowercase}
\fi
% use upquote if available, for straight quotes in verbatim environments
\IfFileExists{upquote.sty}{\usepackage{upquote}}{}
% use microtype if available
\IfFileExists{microtype.sty}{%
\usepackage{microtype}
\UseMicrotypeSet[protrusion]{basicmath} % disable protrusion for tt fonts
}{}
\usepackage{hyperref}
\hypersetup{unicode=true,
            pdftitle={Rose's Thesis\ldots{}},
            pdfauthor={Student number: 2417024},
            pdfborder={0 0 0},
            breaklinks=true}
\urlstyle{same}  % don't use monospace font for urls
\usepackage{natbib}
\bibliographystyle{BES.bst}
\usepackage{longtable,booktabs}
\usepackage{graphicx,grffile}
\makeatletter
\def\maxwidth{\ifdim\Gin@nat@width>\linewidth\linewidth\else\Gin@nat@width\fi}
\def\maxheight{\ifdim\Gin@nat@height>\textheight\textheight\else\Gin@nat@height\fi}
\makeatother
% Scale images if necessary, so that they will not overflow the page
% margins by default, and it is still possible to overwrite the defaults
% using explicit options in \includegraphics[width, height, ...]{}
\setkeys{Gin}{width=\maxwidth,height=\maxheight,keepaspectratio}
\IfFileExists{parskip.sty}{%
\usepackage{parskip}
}{% else
\setlength{\parindent}{0pt}
\setlength{\parskip}{6pt plus 2pt minus 1pt}
}
\setlength{\emergencystretch}{3em}  % prevent overfull lines
\providecommand{\tightlist}{%
  \setlength{\itemsep}{0pt}\setlength{\parskip}{0pt}}
\setcounter{secnumdepth}{5}
% Redefines (sub)paragraphs to behave more like sections
\ifx\paragraph\undefined\else
\let\oldparagraph\paragraph
\renewcommand{\paragraph}[1]{\oldparagraph{#1}\mbox{}}
\fi
\ifx\subparagraph\undefined\else
\let\oldsubparagraph\subparagraph
\renewcommand{\subparagraph}[1]{\oldsubparagraph{#1}\mbox{}}
\fi

%%% Use protect on footnotes to avoid problems with footnotes in titles
\let\rmarkdownfootnote\footnote%
\def\footnote{\protect\rmarkdownfootnote}

%%% Change title format to be more compact
\usepackage{titling}

% Create subtitle command for use in maketitle
\providecommand{\subtitle}[1]{
  \posttitle{
    \begin{center}\large#1\end{center}
    }
}

\setlength{\droptitle}{-2em}

  \title{Rose's Thesis\ldots{}}
    \pretitle{\vspace{\droptitle}\centering\huge}
  \posttitle{\par}
    \author{Student number: 2417024}
    \preauthor{\centering\large\emph}
  \postauthor{\par}
      \predate{\centering\large\emph}
  \postdate{\par}
    \date{Last edited: November 18 2019}

\usepackage{booktabs, amsthm, setspace}
\usepackage[left]{lineno}
\linenumbers
\doublespacing
\setmainfont{Arial}
\setcounter{secnumdepth}{1}

\usepackage[nottoc]{tocbibind}
\bibpunct{(}{)}{;}{a}{}{;}
\setlength{\bibsep}{1em}
\setlength{\bibhang}{1em}
\renewcommand{\bibname}{References}

\makeatletter
\def\thm@space@setup{%
  \thm@preskip=8pt plus 2pt minus 4pt
  \thm@postskip=\thm@preskip
}
\makeatother

\begin{document}
\maketitle

{
\setcounter{tocdepth}{1}
\tableofcontents
}
\hypertarget{intro}{%
\chapter{Introduction}\label{intro}}

\begin{quote}
Would some images of genome duplication be helpful?
\end{quote}

Whole-genome duplication (polyploidisation) is an extreme mutation which occurs due to errors during cell division; such as non-disjunction where sister chromatids are not separated during anaphase.
Failed disjunction leads to all the genetic material ending up in one of the two daughter cells; essentially doubling the genome within that cell.
When this occurs during gametogenesis, it can lead to whole organisms with more than two complete sets of chromosomes (polyploids).
According to \citet{REFNEEDED}, the estimated rate of polyploidisation is one order of magnitude higher than that of standard mutation (10\textsuperscript{-5}).
In humans, it coincides with around 5\% of miscarriages \citep{Creasy1976}.
Moreover, while there have been some live birth cases of triploid human infants, they usually die within hours.
In irregular cases, they survive for several months, but none have survived more than six years \citep{Hashimoto2018}.

\begin{quote}
Newer data for miscarriage rate?
\end{quote}

No polyploid mammal species has ever been found \citep{Svartman2005}.
The same is also usually true for birds \citep{REFNEEDED}.
However, while in these taxa, genome-doubling is not well tolerated, for amphibians, fish, fungi, reptiles, and plants, the story is quite different.
Amongst these, polyploidy is exceptionally well tolerated in plants.
\citet{Jiao2011} found that genome-doubling is a ubiquitous feature in the lineages of almost all flowering plants (angiosperms), and \citet{Wood2009} also found that it has occurred even more recently (post genera formation) for 35\% of all vascular plants.
Plant surveys of natural populations over the last century further support this, reporting mean polyploid frequencies of 25\% \citep{REFNEEDED}.

\hypertarget{the-role-of-genome-doubling-in-plant-diversification}{%
\section{The Role of Genome-Doubling in Plant Diversification}\label{the-role-of-genome-doubling-in-plant-diversification}}

Genome-doubling is thought to be fundamental in the diversification of plant species; having been found to coincide with 15\% of angiosperm speciation events, and 31\% in ferns \citep{Wood2009}.
Furthermore, \citet{Jiao2011} showed that whole-genome duplication could even be linked with major biological innovations that have lead to diversification of the plant phyla.
Their phylogenetic analysis brought to light two ancient groups of duplication events:
the first grouping of duplications (around 319 millennia ago) influenced the diversification of regulatory genes that were integral to seed development and the more recent grouping (around 192 millennia ago) affected genes that enabled flower development.
The authors appropriately conclude that these events must have played a significant role in the rise of seed plants, as well as in the later diversification of angiosperms.
Seeds and flowers were both crucial innovations that allowed plants to transition from an aquatic life-cycle to a terrestrial one by removing their dependence on water for reproduction --- an event, broadly considered as one of the most significant in the history of our planet.

The full story, however, is not quite that clear-cut.
According to \citet{Mayrose2011}, genome-doubling significantly increases speciation rates of diploids, but new polyploid lines do not further speciate by that same mechanism.
So, polyploid speciation rates are far smaller in comparison.
Furthermore, they found polyploid extinction rates to be far higher than those of diploids because polyploids are clumped around the tips of phylogenies.
This Likelihood-based analysis of vascular plants provided clear, and quantitative, support for the traditionally popular view that polyploidy most often leads to evolutionary dead ends.

\begin{quote}
First proposed by? Stebbins 71?
\end{quote}

So, what is it then that allows this extreme mutation to succeed and become fixated?
What scenarios and mechanisms combine to allow rare polyploid lines to succeed?
Are the majority of the polyploids we see in natural populations doomed to extinction?
Furthermore, can we predict which conditions will lead to their survival?

Despite having been studied for over a century, the factors that drive the success of polyploid establishment in the face of high extinction rates are still unclear.
\citet{Arrigo2012} conclude that polyploids tend to become extinct at the establishment phase due to reproductive disadvantages such as triploid sterility, or limited mate-choice; the latter occurring via diploid pollen-swamping, or delayed flowering.
On the other hand, polyploids are frequently linked with distinct traits such as ``gigas effects'', which describes their increased size and plant organs.
Additionally, benefits of polyploidy can include enhanced capabilities for buffering of deleterious mutation (due to increased heterozygosity), hybrid vigour (heterosis), and --- somewhat contentiously --- a reversal of selfing inhibition \citep{Woodhouse2009, Ramsey2014}.
These traits are thought to overcome the reproductive disadvantages of polyploidy and, in those rare cases, make this mutation key to the invasive and adaptive potential of plants. Let us examine these mechanisms in more detail:

\hypertarget{benefits}{%
\section{Benefits}\label{benefits}}

\hypertarget{gigas}{%
\subsection{Gigas-effects}\label{gigas}}

\hypertarget{buffering}{%
\subsection{Genetic buffering}\label{buffering}}

\hypertarget{vigour}{%
\subsection{Hybrid Vigour}\label{vigour}}

\hypertarget{costs}{%
\section{Costs}\label{costs}}

\begin{quote}
Are they realistic? How do they work? Why/when are they important? Consider the conditions.
\end{quote}

\hypertarget{pollen-swamping}{%
\subsection{Diploid Pollen-Swamping}\label{pollen-swamping}}

\hypertarget{flowering}{%
\subsection{Delayed Flowering}\label{flowering}}

\hypertarget{sterility}{%
\subsection{Triploid Sterility}\label{sterility}}

\begin{quote}
So what are the core mechanisms?
\end{quote}

\begin{quote}
Or the most suspicious?
\end{quote}

\hypertarget{how-do-these-mechanisms-link}{%
\section{How do these mechanisms link?}\label{how-do-these-mechanisms-link}}

\begin{quote}
Does limited mate-choice set the scene for the evolution of selfing vs out-crossing?
\end{quote}

\begin{quote}
Delayed flowering (cost) is associated with gigas-effects of increased size (benefit). TRADE-OFF.
\end{quote}

\hypertarget{aims}{%
\section{Aims}\label{aims}}

Based on all this pondering - what am I actually going to examine?

\bibliography{book.bib,packages.bib}


\end{document}
